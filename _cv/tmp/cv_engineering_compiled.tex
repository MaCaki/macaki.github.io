\documentclass[margin,line,pifont,palatino,courier]{res}

\usepackage{pifont}
\usepackage{hyperref}
\usepackage[latin1] { inputenc}
\textheight=9.0in
\addtolength{\topmargin}{-0.5in}


\usepackage{fancyhdr}
\pagestyle{fancy}
\renewcommand{\headrulewidth}{0pt}
\fancyhf{}
\rfoot{ {\footnotesize Curriculum Vitae, Matthew Cadier Kim, \thepage} }


\newenvironment{list1}{
  \begin{list}{\label{ } }{
      \setlength{\itemsep}{0in}
      \setlength{\parsep}{0in} \setlength{\parskip}{0in}
      \setlength{\topsep}{0in} \setlength{\partopsep}{0in}
      \setlength{\leftmargin}{0.0in} } }{\end{list} }
\newenvironment{list2}{
  \begin{list}{$\bullet$}{
      \setlength{\itemsep}{0in}
      \setlength{\parsep}{0in} \setlength{\parskip}{0in}
      \setlength{\topsep}{0in} \setlength{\partopsep}{0in}
      \setlength{\leftmargin}{0.2in} } }{\end{list} }

\begin{document}

\name{Matthew Cadier Kim \vspace*{.1in} }


\begin{resume}


%%%%%%%%%    Contact information   %%%%%%%%%%%


\begin{flushright}
{\small
Thorton Hall 912, 1600 Holloway Ave, San Francisco, CA 94132 \\
1 (415) 767-8984\\
% \url{thecity.sfsu.edu/~mcadier} \\
  \verb+m.cadier.kim@gmail.com+
}
\end{flushright}
 
\section{\sc  About Me}
    
      
          
            I build machine vision software for medical applications and backends for image storage. I also do research in combinatorial optimization, matroid theory and submodular functions, and I like to think about combinatorially inspired algorithms.  I am also interested in transfer learning and convolutional neural networks for training on medical data.

          
      
          
      
          
      
          
      


\section{\sc  Work History}
  
  
    \employer{ UCSF Proctor Foundation }
    \title{ Machine Vision Engineer   }
    \location{  San Francisco, CA   }
    \dates{ May, 2015 - present }

    \begin{position}
    Funded by a grand, I investigated and developed a pipeline and machine learning algorithms to classify images of eyelids. I worked with the PI to define the direction of the project and was the principal developer of the grading application.
    \end{position}
  
    \employer{ San Francisco State University }
    \title{ Lecturer in Mathematics  }
    \location{  San Francisco, CA   }
    \dates{ September, 2014 - December, 2015 }

    \begin{position}
    I was the principal instructor for several sections of college algebra and precalculus.  I helped develop and administer a large online calculus course for 200+ students.
    \end{position}
  
    \employer{ Argyle Data }
    \title{ Data Scientist  }
    \location{  San Mateo, CA   }
    \dates{ June, 2014 - February, 2015 }

    \begin{position}
    I prototyped statistical machine learning algorithms for time series analysis and fraud detection in network data as well as implemented production versions in Java.
    \end{position}
  

\section{\sc Languages/ \\ Tools}

  \begin{tabular}{@{}p{6in}p{3in}}

     English  French (Fluent)  Java  Python  Matlab  Julia  C++ 
    \\working with: \ 
     AWS  OpenCV  Scikit-learn 

  \end{tabular}


\section{\sc Papers}
  \begin{tabular}{@{}p{5in}}
    \begin{list1}
      
      
        
            \item {\em A Characterization of Generalized Permutohedra for the Classical Reflection Groups},  2015, (thesis)  
        
      \vspace{.2em}
      
        
      \vspace{.2em}
      
        
      \vspace{.2em}
      
        
            \item {\em Discriminating Eyelids with Trachomatous Inflammation - Follicular},  with Dr. Travis Porco and Dr. Kazunori Okada,  2015, (in progress)  
        
      \vspace{.2em}
      
    \end{list1}
  \end{tabular}


\section{\sc Education} 
  %\vspace*{-.1in}
  \begin{list1}
    
    
        \item  {\bf SFSU:}\textbf{ MA }, Mathematics, 2015

          
    
        \item  {\bf SFSU:}\textbf{ BA }, Philosophy, Mathematics, Computer Science, 2012

          
    
        \item  {\bf University of Paris 1 Pantheon la Sorbonne:}\textbf{ Visiting Student }, Philosophy, Logic, Mathematics, 2009-2011

          
    

  \end{list1}



\end{resume}



\end{document}

