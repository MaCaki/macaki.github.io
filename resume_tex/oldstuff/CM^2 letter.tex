% Cover letter using letter.cls
\documentclass[11pt]{letter} % default is 10 pt
%\usepackage{helvetica} % uses helvetica postscript font (download helvetica.sty)
\usepackage{newcent}   % uses new century schoolbook postscript font 
% the following commands control the margins:
\topmargin=-1in    % Make letterhead start about 1 inch from top of page
\textheight=8.5in  % text height can be bigger for a longer letter
\oddsidemargin=0pt % leftmargin is 1 inch
\textwidth=6.5in   % textwidth of 6.5in leaves 1 inch for right margin

\begin{document}


\signature{Matthew Kim}                  % name for signature 
\longindentation=0pt                       % needed to get closing flush left
%\let\raggedleft\raggedright                % needed to get date flush left
 
\begin{letter}{(CM)$^2$ Graduate Student Fellowship for the Academic year 2012/2013\\
 Letter of Motivation} 

\begin{center}
\large\bf Matthew Kim\\
1712 Quesada Ave\\ San Francisco, CA 94124\\ (949) 813-5113
\end{center} 
\vfill % forces letterhead to top of page

\opening{} 



While many of my colleagues are at the beginning of their career as academic mathematicians, I�ll be the first to admit that I am still at the beginning of mine as a math student.   Having a degree in philosophy, I spent many years studying more literary forms of knowledge.  However I don't see my current interests as incongruous with my former studies.  It was thanks to philosophy, and in particular, through an inspired logic teacher at the University of Paris 1 that I discovered math. One of the benefits of having spent time at this institution was a completely paradigm-shifting exposure to an academic culture that really engaged mathematics as a struggle with the limits of human reason. It considered dimensions outside the technical discourse: historical, cultural, linguistic and philosophical. 

	 Going back far enough, almost all of the great �literary� thinkers were polymaths, gorging on all types and flavors of science, but unified in their approach by a careful, rational method.  The classic example is Rene Descartes, who every school child has to thank for the planar coordinate system. However, in his Discourse on the Method, published with his work on geometry, he also sets a precedent in mental organization.  Possession of reason is one thing, but employing that to the effective discovery of truth is another.   From the simplest, most certain truths, we can build greater ones, avoiding any unverified premise.
Inspired by this classical disposition to universal knowledge, I started seriously questioning my assumption that mathematics and the hard sciences were out of my grasp at this point. Equipped with the method, I believe, there is no domain of human knowledge that is closed off to the rigorous mind.  After spending some time studying the history and language of mathematics, I decided that I needed some first hand experience in the field.  

    Studying philosophy and logic has not given me universal knowledge, but it has trained me in this universal method of learning. The foundation of every worthwhile intellectual venture is no more philosophy than it is math. It is the rational optimism with which we approach the problem.   So though this is only my second academic year studying mathematics seriously, armed with this method, I have rarely found myself at a disadvantage vis a vis those with more material under their belt.  It is mathematic's accessibility to any rational, patient mind that I would like to exploit for myself and show others; to help others see past the often imposing and intimidating first impression of science and begin to think of it as something that they can take part in.    

However to get to this point so quickly I had to take an accelerated course has left me little time for more in depth research.  Heavy course loads, and now, heavy teaching loads prevent me from rounding out many mathematical interests.   I envision a (CM)$^2$ fellowship as just this opportunity to start laying down a more lasting mathematical foundation. I will have finished most of the unit and course requirements for the MA by next spring, so am hoping to spend more time broadening and deepening my mathematical activities.      

    The teaching responsibilities of (CM)$^2$  fellows are similar to ones I have already met in previously held tutoring positions.  As a GEAR-UP tutor, I was placed in a (CM)$^2$  teacher�s class and sometimes worked alongside a (CM)$^2$  fellow. At that point my responsibilities were simply to aid any students who needed it, but after a year of working in San Francisco Public schools in math and science classrooms, I was able to observe many of the issues that (CM)$^2$  and similar programs are addressing: boredom for those students at the top, and frustration for those at the bottom due to a system that seems more concerned about preserving the curriculum rather than the students who have to learn it.  
    
    In addition to my GTA responsibilities at SFSU, I currently work with an after school tutoring program in Oakland, OSMO, where the philosophy and goals are in step with the ideas behind Math Circle and (CM)$^2$ .  We give motivated high school students a clean and professional space to work after school and interact with university students as well as get involved in weekly workshops that break down traditional barriers to technology and engineering.  Students build Android apps, learn about 3D modeling, and are allowed to contextualize often inaccessible scientific information.
    
    I am eager to turn these experiences with high school students into portable curriculum.  At OSMO, we have been developing resources for a more inquiry based tutoring style. I like to think that these practices can be adapted to the actual class room. Getting back into a high school class room now after working
    with college freshman would give me a completely different perspective than what I initially had last year.  I would also have a greater sense of urgency in trying
    to affect change in the classroom culture, and in the language that students use when talking about their struggles with mathematics. Too often, our potential is hijacked by a discourse that we adopt from our surroundings. What I hope to rediscover as a (CM)$^2$  fellow is a more direct relationship with the students and also the time to devote to what I am most passionate about in education: trying to change how our culture views mathematics and talks about
    math education.  
    
     I am heartened to see that past and current fellows come from very diverse backgrounds, showing that there is no one road to mathematics. 
    Likewise there is no easy road to mathematics.  But many of us in the master's program here at SFSU have chosen to take that journey in spite of experiences
    in elementary education. I had at one point completely written off all sciences.  Reflection on the change of spirit that led me to math I think makes me, and 
    many of my colleagues particularly well equipped to incite interest in mathematics beyond the usual groups traditionally associated with math success. First hand
    experience with the frustrations and turn offs of the elementary and high school math curriculum makes us that much more prepared to help young people avoid them. 
    
      
      
      \newpage
      {\bf Planned Reseach}
      
      Polyhedron are generalizations of the regions described by simple linear inequalities that we all learn about in basic algebra.  We simply have 
a "shaded" region on a cartesian plane or in a 3 dimensional space.  This simple idea can be extended to higher dimensions by just adding more variables and ignoring our difficulty in graphing the resulting region.  We would still interpret the region as a set of points that satisfy our system of linear
inequalities.  This is a very simple and effective way of modeling many real life systems. Linear optimization, which studies how to maximize 
functions over these sets, calls them "feasible regions."  A {\it polytope} is simply a polyhedron that is closed.  In two dimensions this just means
a closed shape with straight sides:  a polygon.  And this generalizes easily to higher dimensions.     

	In any dimension we will still have vertices : one-dimensional points that make up the extreme "corners" of a polytope.  In fact, we
	can think of polytopes as just sets of points, about which we cover a tight surface that encloses everything between the points, like
	setting up a tent in space.  Some points will be connected by lines, others may be on the same plane, and other will be on different
	sides of the polytope.  We can turn these intuitive spacial perceptions into generalizations for higher dimensions. If instead of viewing a polytope as a spacial object and instead we look at it as this system of relations between vertices, we can categorize it according to it's {\it combinatorial type.}
	
	One of the many attractions of polytopes is this ability to simultaneously embed so many types of structures that we are interested in exploring. 
	Viewing them through the lens of combinatorial types, we see that we can morph polytopes in some ways such that the relationships
	between the vertices are unchanged.  Take a triangle and stretch it.  We still have three vertices connected to each other.  However
	in the case of the square, if we bend it in the wrong way, we can end up with some vertices changing position relative to the others so that 
	when we stretch a surface around them, we end up with a triangle.   
	The  {\it realization space}
      of a polytope describes just those movements that will preserve the combinatorial structure of the vertices.  Much work has already been done
      to categorize combinatorial types for  polytopes in certain dimensions and their associated realizations spaces.  We encode the movements that we make 	to 
      the vertices using matrices. The realization space then is the set of matrices corresponding to the "allowable" movements.  
      
      For the case of 3 -dimensional polytopes, we already have some nice results.  However the situation gets much worse for higher dimensions.
      A 1995 paper shows that the realization space, or the attempt to encode the relations between the vertices of a 4-dimensional polytope, can
      be arbitrarily complicated, whereas in the 3-dimensional case we can transform the vertices of any polytope so that
      all the coordinates are integers and still "realize" the same combinatorial structure.   
      
      However there have been many results in this effort
      to catalog the combinatorial types of polytopes, and I plan to work with Dr. Gubeladze to try and extend them to polytopal complexes.  These
      are simply two or more polytopes glued together along faces that match up.  These structures can become more complicated than 
      just single polytopes, but they're vertices are still related to each other in a way that we can categorize.     We hope to start
      developing programs soon that will describe the realization space of polytopes and polytopal complexes by using multivariate polynomials.
      This means that there is a fascinating interaction between polytopes and algebraic geometry, which studies the solutions spaces of polynomials. 
        The rest of this semester will
      be spent participating in a reading group on Dr. Gubeladze's and Dr. Bruns book and familiarizing myself with the fundamentals of polytopes
      and algebraic geometry in order to map out the next steps in this project. \\
      
{\bf Future plans }


After completing my MA, I plan on continuing to study mathematics in a PhD program. Doctoral studies
Though graduate studies have so far proved to be quite
a commitment, after six years of continuous academic work my stamina remains intact, and I remain convinced that this is the my ideal path. I very much enjoy sharing my energies between 
teaching and researching and am so far very satisfied with what SFSU has allowed me to do.  

 In five years I will be close to obtaining a PhD in 
mathematics.   I see myself still engaged in educational initiatives and the ongoing effort to demystify the field for young students and new initiates.  
These past two semesters I have been involved with Dr. Arek Goetz's online calculus classes, where we are constantly running into novel issues
associated with distance learning and online platforms.  The role of the university is evolving due to large scale changes in how information is communicated. However the nature of education is still shaped by the ideas and actions of teachers and learners.  I hope to continue to participate
in shaping what the future of education will look like.   I am currently learning web programming, and working with OSMO to develop a catalog
of online resources that are in line with our pedagogy.   I hope that in the coming years I will produce web resources for elementary and
high school math.  In this way I hope to do my part in changing the culture of how non-mathematicians see math.  

This quickly evolving state of education and communication makes predictions about what exactly my role will be in the future.   But if the last
two years are any indication, I am quite excited about whatever comes next.   Two years ago, I would have never imagined that I would
 be doing graduate work in mathematics and be at the level that I am at.  This contributes to my confidence that mathematics is truly 
available to anyone with an open mind and lots of diligence.  So I see my personal development over the next five years as a continuation of 
slowly mapping out more of the vast mathematical territory around me.  I hope to be active in my field of specialization, but also to have filled
in gaps in my mathematical and scientific culture.  However pretending to know what I'll be doing, or even what I'll {\it want} to be doing
in five years is quite hopeless once I realize that just five years ago I had just finished high school, and thought that I'd never take 
another math class again in my life.   

Despite this unpredictability, I hope that my record of experience in education, and my academic progress shows that my future is worth investing in. 
    \closing{Sincerely,}
   
 
 
 				% Enclosures

\end{letter}
 
 
 \newpage

\end{document}
