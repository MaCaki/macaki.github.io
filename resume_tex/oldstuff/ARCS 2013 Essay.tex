% Cover letter using letter.cls
\documentclass[11pt]{letter} % default is 10 pt
%\usepackage{helvetica} % uses helvetica postscript font (download helvetica.sty)
\usepackage{newcent}   % uses new century schoolbook postscript font 
% the following commands control the margins:
\topmargin=-1in    % Make letterhead start about 1 inch from top of page
\textheight=8.5in  % text height can be bigger for a longer letter
\oddsidemargin=0pt % leftmargin is 1 inch
\textwidth=6.5in   % textwidth of 6.5in leaves 1 inch for right margin
\usepackage{setspace}
\onehalfspacing
\renewcommand*{\today}{April 17, 2013}





\begin{document}


\signature{Matthew Cadier Kim}
\setlength\parindent{24pt}                  % name for signature 
%\longindentation=0pt                       % needed to get closing flush left
%\let\raggedleft\raggedright                % needed to get date flush left
 
\begin{letter}{\bf ARCS Scholarship Letter of Motivation} 

\begin{center}
\large{\bf Matthew Kim}\\
1712 Quesada Ave\\ San Francisco, CA 94124\\ (949) 813-5113
\end{center} 
\vfill % forces letterhead to top of page

\opening{} 



%Professional Goals: 
\quad \quad I am a first year graduate student in mathematics. My initial introduction to mathematics was as an undergraduate student of logic at the University of Paris 1.  Though  my mathematical interests have moved far beyond this initial study of logic, my engagement in the field is still motivated by the same philosophy: I hope that as mathematician I may answer very precise questions arising from problems given in the empirical sciences while simultaneously confronting more fundamental questions about the nature of scientific knowledge.
 
Early on I formed a strong interest in algebra and geometry and have commenced a research project under Dr. Gubeladze's supervision. The project will be a novel fusion of polytopal theory and algebra.  Polytopes provided the Greeks with one of the first examples of how to generalize thoughts about crude drawings and representations into purely abstract mathematical objects.  Rather than elaborating the cult of Platonic Ideals however, the past 50 years have seen a resurgence of relevance of discrete geometry. Polytopes have become ubiquitous throughout physical and social sciences as linear modeling become easier with advances in computing.  Using the language of linear algebra, it becomes very easy to characterize sets of feasible solutions to a linear optimization problem as a polyhedra, or a polytope. 

In the mid 1990's J\"urgen Richter-Gebert built on a result of Nikolai Mnev to start a line of research that would algebraically characterize the combinatorial types of single polytopes.  Combinatorial types encode information about a polytope's vertices and faces.  With Dr. Gubeladze I hope to extend this to the algebraic characterization of higher dimensional objects that use polytopes as elementary building blocks: polytopal complexes. Potential applications include fields that have already incorporated the language of polytopes from robotics and linear programming to core mathematical disciplines such as 
algebraic geometry. 

Yet my ability to commit to this research is threatened by financial instability. My father's position as a civilian engineer working for the Navy was recently reclassified as non-essential, meaning his salary will suffer large reductions this year under the 2013 Federal budget sequestration. Additionally, my mother is retiring and I will not be receiving the support that I had as an undergraduate. Therefore I will be entirely dependent on the teaching posts available through the math department. Though this has been an invaluable experience, the responsibilities and work associated with the posts have made it extremely difficult to find time for serious research. 

    I have not taken a break from school to build up a savings, nor do I currently benefit from a grant or scholarship. 
Seeking employment outside the department will exclude me from the community of graduate students. However given the cost of living in the Bay area such alienating options may be unavoidable.

This coming year will be a crucial moment in my academic career.   I intend to apply to doctoral programs by the end of next year where I expect to expand the applications of my current project in polytopal theory to linear optimization and commutative algebra.  I anticipate that the usefulness of discrete geometry will give me career options in both industry and academia. 

The fact that I was able to advance from the beginning of the calculus sequence to novel research in polytopal theory within four semesters gives me reason to be confident in my potential as a mathematician.  My record also shows a commitment to STEM through education community oriented mentor programs. I have come to the point where I believe my efforts warrant not only financial support, but the recognition that my future as a researcher is worth investing in. 







    \closing{Sincerely,}
   
 
 
 				% Enclosures

\end{letter}
 
 
 \newpage

\end{document}